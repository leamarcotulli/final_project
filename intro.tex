\section*{Introduction}

\hspace{12ex}High mass stars are like rock stars: they live fast, die young, and leave a beautiful corpse. In more scientific terms, these stars have an initial high mass, therefore consume their fuel faster than a lower mass star ($\sim 10^7-10^8$ years) and end their lifes with a violent explosion called supernova. Depending on the mass of the star, we can have different kind of supenovae. The case we are interested in in this short paper, are the so-called core collapse supernovae, whose progenitors have a mass $8 \Msun \leq M \leq 50 \Msun$.

\hspace{12ex}Like every other star, the first thermonuclear reaction that onsets is the hydrogen burning. When all the hydrogen in the core is exhausted, the star starts contracting due to gravity. The temperature in the core increases until helium fusion can occurr. Stars with $M\geq8\Msun$ are massive enough to reach core temperatures that onset burning of heavier elements. Every time one source of fuel is exhausted, the core contracts under gravity until reaching the right temperature to trigger another element burning.

\hspace{12ex}The process ceases when elementary fusion realeases the minimum amount of energy allowed, which in turn depends on the binding energy of the various elements. The last elements produced by thermonuclear fusion are nickel and iron, which have the maximum binding energy per nucleon. 

\hspace{12ex}At this stage, the pressure at the core is huge and, since there is no fusion to support gravitational contraction, the only force acting against the collapse is the degeneracy pressure of electrons. When the core exceeds the Chandrasekhar limit ($M \geq 1.4 \Msun$), degeneracy pressure cannot anymore support the gravitational force: the core collapses ($v\sim 7\times 10^9 cm/s$) and heats up. In this phase, among others, $\gamma$-rays and neutrinos are produced. Neutrinos can escape the dense core, carrying away some of the energy and accelerating the collapse. This is when the supernova explosion begins.   

\hspace{12ex}As the shock travels through the stellar medium, new elements are formed. Also, the presence of neutrinos alters the composition of the material.
The problem I will try to adress through this project is understanding the production of $^{97}$Tc, $^{98}$Tc and $^{99}$Tc. Tc is not a stable element, so if we detect this element in the supernova, it needs to be produced in loco. How? That is what I will try to figure out. 

\hspace{12ex}In the next sessions I will describe all instructions necessary to accomplish the task. All files and commands needed in order to reproduce the results will be provided. 



